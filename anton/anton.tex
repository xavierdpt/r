\documentclass[]{article}
\usepackage[margin=1cm]{geometry}
\usepackage[T1]{fontenc}
\usepackage[utf8]{inputenc}
\usepackage{amsmath,amsfonts,amssymb}
\parindent0pt
\newcommand{\fgraph}[1]{\fbox{\small#1}}
\newcommand{\pgraph}[1]{\par$\bullet$ \textbf{#1}\par}
\newcommand{\D}{\hskip1pt{\mathrm{d}}}
\newcommand{\mrm}{\medskip\hrule\medskip}
\pagestyle{empty}
\begin{document}
\section{Functions}
\subsection{Functions and The Analysis of Graphical Information}
\subsection{Properties of Functions}
\subsection{Graphic Functions on Calculators and Computers; Computer Algebra Systems}
\subsection{New Functions from Old}
\subsection{Mathematical Models; Linear Models}
\subsection{Families of Functions}
\subsection{Parametric Equations}
\section{Limits and Continuity}
\subsection{Limits (An Intuitive Introduction)}
\fgraph{The Tangent Line, Area, and Velocity Problems}
\fgraph{Tangent Lines and Limits}
\fgraph{Instantenous Velocity and Limits}
\fgraph{Limits}
\fgraph{Numerical Pitfalls}
\fgraph{One-Sided Limits}
\fgraph{The Relationship Between One-Sided and Two-Sided Limits}
\fgraph{A First Look at Continuity}
\fgraph{Infinite Limits and Vertical Asymptotes}
\fgraph{Limits at Infinity and Horizontal Asymptotes}
\fgraph{How Limits at Infinity Can Fail to Exist}
\subsection{Limits (Computational Techniques)}
\fgraph{Some Basic Limits}
\fgraph{Limits of Polynomials as $x\to a$}
\fgraph{Limits of $x^n$ as $x\to+\infty$ or $x\to-\infty$}
\fgraph{Limits of Polynomials as $x\to+\infty$ or $x\to-\infty$}
\fgraph{Limits of Rational Functios as $x\to a$}
\fgraph{Limits of Rational Functions as $x\to+\infty$ or $x\to-\infty$}
\fgraph{A Quick Method for Finding Limits of Rational Functions as $x\to+\infty$ or $x\to-\infty$}
\pgraph{Limits Involving Radicals}
\mrm
$$
\lim_{x\to+\infty}\sqrt[3]{\frac{3x+5}{6x-8}}
=
\sqrt[3]{\lim_{x\to+\infty}\frac{3x+5}{6x-8}}
=
\frac{1}{\sqrt[3]2}
$$
\mrm
$$
\lim_{x\to+\infty}\frac{\sqrt{x^2+2}}{3x-6}
=
\lim_{x\to+\infty}\frac{\sqrt{x^2+2}/|x|}{(3x-6)/|x|}
\overset{x>0}{=}
\lim_{x\to+\infty}\frac{\sqrt{x^2+2}/\sqrt{x^2}}{(3x-6)/x}
=
\lim_{x\to+\infty}\frac{\sqrt{1+2/x^2}}{3-6/x}
=
\frac{\sqrt{\displaystyle\lim_{x\to+\infty}1+2/x^2}}{\displaystyle\lim_{x\to+\infty}3-6/x}
=
\frac{1}{3}
$$
\mrm
$$
\lim_{x\to-\infty}\frac{\sqrt{x^2+2}}{3x-6}
=
\lim_{x\to-\infty}\frac{\sqrt{x^2+2}/|x|}{(3x-6)/|x|}
\overset{x<0}{=}
\lim_{x\to-\infty}\frac{\sqrt{x^2+2}/\sqrt{x^2}}{(3x-6)/(-x)}
=
\lim_{x\to-\infty}\frac{\sqrt{1+2/x^2}}{-3+6/x}
=
\frac{\sqrt{\displaystyle\lim_{x\to-\infty}1+2/x^2}}{\displaystyle\lim_{x\to-\infty}-3+6/x}
=
-\frac{1}{3}
$$
\mrm
\fgraph{Limits of Functions Defined Piecewise}
\subsection{Limits (Discussed More Rigorously)}
\fgraph{Definition of a Limit}
\fgraph{The Value of $\delta$ Is Not Unique}
\fgraph{Limits as $x\to+\infty$ or $x\to-\infty$}
\fgraph{Infinite Limits}
\subsection{Continuity}
\fgraph{Definition of Continuity}
\fgraph{Continuity in Applications}
\fgraph{Continuity of Polynomials}
\fgraph{Some Properties of Continuous Functions}
\fgraph{Continuity of Rational Functions}
\fgraph{Continuity of Compositions}
\fgraph{Continuity from the Left and from the Right}
\fgraph{The Intermediate-Value Theorem}
\fgraph{Approximating Roots Using the Intermediate Value Theorem}
\fgraph{Approximating Roots by Zooming with a Graphic Utility}
\subsection{Limits and Continuity of Trigonometric Functions}
\fgraph{Continuty of Trigonometric Functions}
\fgraph{Obtaning Limits by Squeezing}
%The Derivative%
\section{The Derivative}
\fgraph{Slope of a Tangent Line}
\fgraph{Average Versus Instantaneous Velocity}
\fgraph{Average and Instantaneous Rate of Change}
\subsection{Tangent Lines and Rates of Change}
\fgraph{Tangent Lines Defined Precisely}
\fgraph{Slopes of Tangent Lines by Zooming}
\subsection{The Derivative}
\fgraph{The Derivative}
\fgraph{Differentiability}
\fgraph{Relationship Between Differentiability and Continuity}
\fgraph{Derivative Notation}
\fgraph{Other Notations}
\fgraph{Derivatives at the Endpoints of an Interval}
\subsection{Techniques of Differentiation}
\fgraph{Derivative of a Constant}
\fgraph{Derivative of $x$ to a Power}
\fgraph{Derivative of a Constant Times a Function}
\fgraph{Derivatives of Sums and Differences}
\fgraph{Derivative of a Product}
\fgraph{Derivative of a Quotient}
\fgraph{Derivative of a Reciprocal}
\fgraph{The Power Rule for Integer Exponents}
\fgraph{Higher Derivatives}
\subsection{Derivatives of Trigonometric Functions}
\fgraph{Derivatives of the Trigonometric Functions}
\subsection{The Chain Rule}
\fgraph{Derivatives of Compositions}
\fgraph{Generalized Derivative Formulas}
\fgraph{An Alternative Approach to Using the Chain Rule}
\fgraph{Differentiating Using Computer Algebra Systems}
\subsection{Local Linear Approximation; Diffferentials}
\fgraph{Increments}
\fgraph{Differentials}
\fgraph{Local Linear Approximation}
\fgraph{Error in Local Linear Approximations}
\fgraph{Error Propagation in Applications}
\fgraph{Differential Formulas}
\section{Logarithmic and Exponential Functions}
\subsection{Inverse Functions}
\fgraph{Inverse Functions}
\fgraph{Domain and Range of Inverse Functions}
\fgraph{A Method for Finding Inverses}
\fgraph{Existence of Inverse Functions}
\fgraph{Graphs of Inverse Functions}
\fgraph{Increasing or Decreasing Functions Have Inverses}
\fgraph{Restricting Domains to Make Functions Invertible}
\fgraph{Continuity of Inverse Functions}
\fgraph{Differentiability of Inverse Functions}
\fgraph{Graphing Inverse Functions with Graphing Utilities}
\subsection{Logarithmic and Exponential Functions}
\fgraph{Irrational Exponents}
\fgraph{The Family of Exponential Functions}
\fgraph{Logarithms}
\fgraph{Logarithmic Functions}
\fgraph{Solving Equations Involving Exponentials and Logarithms}
\fgraph{Change of Base Formula for Logarithms}
\fgraph{Logarithmic Scales in Science and Engineering}
\fgraph{Exponential and Logarithmic Growth}
\subsection{Implicit Differentiation}
\fgraph{Functions Defined Explicitly and Implicitly}
\fgraph{Graphs of Equations in $x$ and $y$}
\fgraph{Implicit Differentiation}
\fgraph{Differentiability of Functions Defined Implictly}
\fgraph{Derivatives of Rational Powers of $x$}
\fgraph{Derivatives of Inverse Functions}
\subsection{Derivatives of Logarithmic and Exponential Functions}
\fgraph{Derivatives of Logarithmic Functions}
\fgraph{Logarithmic Differentiation}
\fgraph{Derivatives of Irrational Powers of $x$}
\fgraph{Derivatives of Exponential Functions}
\subsection{Derivatives of Inverse Trigonometric Functions}
\fgraph{Inverse Trigonometric Functions}
\fgraph{Evaluating Inverse Trigonometric Functions}
\fgraph{Identities for Inverse Trigonometric Functions}
\fgraph{Derivatives of the Inverse Trigonometric Functions}
\fgraph{Differentiability of the Inverse Trigonometric Functions}
\subsection{Related Rates}
\fgraph{Rates of Changes Using the Chain Rule}
\subsection{L'Hôpital's Rule ; Inderterminate Forms}
\fgraph{Indeterminates Forms of Type 0/0}
\fgraph{L'Hôpital's Rule}
\fgraph{Indeterminate Forms of Type $\infty/\infty$}
\fgraph{Analyzing the Growth of Exponential Functions Using l'Hôpital's Rule}
\fgraph{Indeterminate Forms of Type $0\cdot\infty$}
\fgraph{Indeterminate Forms of Type $\infty-\infty$}
\fgraph{Indeterminate Forms of Type $0^0$, $\infty^0$, $1^\infty$}
\section{Analysis of Functions and Their Graphs}
\subsection{Analysis of Functions I: Increase, Decrease, and Concavity}
\subsection{Analysis of Functions II: Relative Extrema; First and Second Derivative Tests}
\subsection{Analysis of Functions III: Applying Technology and the Tools of Calculus}
\section{Applications of the Derivative}
\subsection{Absolute Maxima and Minima}
\subsection{Applied Maximum and Minimum Problems}
\subsection{Rectilinear Motion (Motion Along a Line)}
\subsection{Newton's Method}
\subsection{Rolle's Theorem; Mean-Value Theorem}
%Integration%
\section{Integration}
\subsection{An Overview of the Area Problem}
\fgraph{Defining Area}
\fgraph{The Rectangle Method for Finding Areas}
\fgraph{The Antiderative Method for Finding Areas}
\subsection{The Indefinite Integral; Integral Curves and Direction Fields}
\fgraph{The Indefinite Integral}
\pgraph{Integration Formulas}
$$\int \D x=x+C$$
$$\int x^r\D x=\frac{x^{r+1}}{r+1}+C$$
$$\int \cos x\D x=\sin x+C$$
$$\int \sin x\D x=-\cos x+C$$
$$\int (\sec x)^2\D x=\tan x+C$$
$$\int (\csc x)^2\D x=-\cot x+C$$
$$\int \sec x\tan x\D x=\sec x+C$$
$$\int \csc x\cot x\D x=-\csc x+C$$
$$\int e^x\D x=e^x+C$$
$$\int b^x\D x=\frac{b^x}{\ln b}+C$$
$$\int \frac{1}{x}\D x=\ln|x|+C$$
\pgraph{Properties of the Indefinite Integral}
$$\frac{\D}{\D x}\left(\int f(x)\D x\right)=f(x)$$
\fgraph{Integral Curves}
\fgraph{Integration from the Viewpoint of Differential Equations}
\fgraph{Direction Fields}
\subsection{Integration by Substitution}
\fgraph{$u$-Substitution}
\fgraph{Integration Using Computer Algebra Systems}
\subsection{Sigma Notation}
\fgraph{Sigma Notation}
\fgraph{Changing the Index of Summation}
\fgraph{Properties of Sigma Notation}
\fgraph{Summation Formulas}
\subsection{The Definite Integral}
\fgraph{A Definition of Area}
\fgraph{The Definite Integral of a Continuous Function}
\fgraph{The Riemann Integral}
\fgraph{Integrability}
\fgraph{Properties of the Definite Integral}
\fgraph{Conditions for Integrability}
\subsection{The Fundamental Theorem of Calculus}
\fgraph{The Fundamental Theorem of Calculus}
\fgraph{The Relationship Between Definite and Indefinite Integrals}
\fgraph{Dummy Variables}
\fgraph{The Mean-Value Thereom for Integrals}
\fgraph{Part 2 of the Fundamental Theorem of Calculus}
\fgraph{Differentiation and Integration are Inverse Processes}
\subsection{Rectilinear Motion Revisited; Average Value}
\fgraph{Finding Position and Velocity by Integration}
\fgraph{Uniformly Accelerated Motion}
\fgraph{The Free-Fall Model}
\fgraph{Integrating Rates of Change}
\fgraph{Displacement in Rectilinear Motion}
\fgraph{Distance Traveled in Rectilinear Motion}
\fgraph{Analyzing the Velocity Versus Time Curve}
\fgraph{Average Value of a Continuous Function}
\fgraph{Average Velocity  Revisited}
\subsection{Evaluating Definite Integrals by Substitution}
\fgraph{Two Methods for Making Substitutions in Definite Integrals}
\subsection{Logarithmic Functions from the Integral Point of View}
\fgraph{The Link Between Natural Logarithms and Integrals}
\fgraph{Approximating $\ln x$ Numerically}
\fgraph{Differentiability and Continuity of $\ln x$ and $e^x$}
\fgraph{The Definition of $e$ revisited}
\fgraph{Functions Defined by Integrals}
\fgraph{Evaluating and Graphing Functions Defined by Integrals}
\fgraph{Integrals with Function as Limits of Integration}
\section{Applications of the Definite Integral in Geometry, Science, and Engineering}
\subsection{Area Between Two Curves}
\subsection{Volumes by Slicing; Disks and Washers}
\subsection{Volumes by Cylindrical Shells}
\subsection{Length of a Plane Curve}
\subsection{Area of a Surface of Revolution}
\subsection{Work}
\subsection{Fluid Pressure and Force}
\subsection{Hyperbolic Functions and Hanging Cables}
\section{Principles of Integral Evaluation}
\subsection{An Overview of Integration Methods}
\fgraph{Methods for Approaching Integration Problems}
\fgraph{A Review of Familiar Integration Formulas}
\subsection{Integration by Parts}
\fgraph{Derivation of the Formula for Integration by Parts}
\fgraph{Integration by Parts for Definite Integrals}
\fgraph{Reduction Formulas}
\subsection{Trigonometric Integrals}
\fgraph{Integrating Powers of Sine and Cosine}
\fgraph{Integrating Products of Sines and Cosines}
\fgraph{Integrating Powers of Tangent and Secant}
\fgraph{Integrating Products of Tangents and Secants}
\fgraph{An Alternative Method for Integrating Powers of Sine, Cosine, Tangent and Secant}
\fgraph{Mercator's Map of the World}
\subsection{Trigonometric Substitutions}
\fgraph{The Method of Trigonometric Substitution}
\fgraph{Integrals Involving $ax^2+bx+c$}
\subsection{Integrating Reational Functions by Partial Fractions}
\fgraph{Partial Fractions}
\fgraph{Finding the Form of a Partial Fraction Decomposition}
\fgraph{Linear Factors}
\fgraph{Quadratic Factors}
\fgraph{Integrating Improper Rational Functions}
\fgraph{Concluding Remarks}
\subsection{Using Tables of Integrals and Computer Algebra Systems}
\fgraph{Integral Tables}
\fgraph{Perfect Matches}
\fgraph{Matches Requiring Substitutions}
\fgraph{Matches Requiring Reduction Formulas}
\fgraph{Matches Requiring Special Substitutions}
\fgraph{Integrating with Computer Algebra Systems}
\fgraph{Computer Algebra Systems can Fail}
\subsection{Numerical Integration; Simpson's Rule}
\fgraph{A Review of Riemann Sum Approximations}
\fgraph{Trapezoidal Approximation}
\fgraph{Comparison of the Midpoint and Trapezoidal Approximations}
\fgraph{Simpson's Rule}
\fgraph{Error Estimates}
\fgraph{A Comparison of the Three Methods}
\subsection{Improper Integrals}
\fgraph{Improper Integrals}
\fgraph{Integrals over Infinite Intervals}
\fgraph{Integrals whose Integrands have Infinite Discontinuities}
\fgraph{The Application of Improper Integrals to Arc Length and Surface Area}
\section{Mathematical Modeling with Differential Equations}
\subsection{First-Order Differential Equations and Applications}
\subsection{Direction Fields; Euler's Method}
\subsection{Modeling with Differential Equations}
\section{Infinite Series}
\subsection{Sequences}
\fgraph{Definition of a Sequence}
\fgraph{Graphs of Sequences}
\fgraph{Limit of a Sequence}
\fgraph{The Squeezing Theorem for Sequences}
\fgraph{Sequences Defined Recursively}
\subsection{Monotone Sequences}
\fgraph{Terminology}
\fgraph{Testing for Monotonicity}
\fgraph{Properties that Hold Eventually}
\fgraph{An Intuitive View of Convergence}
\fgraph{Convergence of Monotone Sequences}
\subsection{Infinite Series}
\fgraph{Sum of Infinite Series}
\fgraph{Geometric Series}
\fgraph{Harmonic Series}
\subsection{Convergence Tests}
\fgraph{The Divergence Test}
\fgraph{Algebraic Properties of Infinite Series}
\fgraph{The Integral Test}
\fgraph{$p$-Series}
\fgraph{Proof of the Integral Test}
\subsection{Taylor and Maclaurin Series}
\fgraph{Local Quadratic Approximations}
\fgraph{MacLaurin Polynomials}
\fgraph{Taylor Polynomials}
\fgraph{Sigma Notation for Taylor and MacLaurin Polynomials}
\fgraph{Taylor and MacLaurin Series}
\subsection{The Comparison, Ratio, and Root Tests}
\fgraph{The Comparison Test}
\fgraph{Using the Comparison Test}
\fgraph{The Limit Comparison Test}
\fgraph{The Ratio Test}
\fgraph{The Root Test}
\subsection{Alternating Series; Conditional Convergence}
\fgraph{Alternating Series}
\fgraph{Approximating Sums of Alternating Series}
\fgraph{Absolute Convergence}
\fgraph{Conditional Convergence}
\fgraph{The Ratio Test for Absolute Convergence}
\fgraph{Summary of Convergence Tests}
\subsection{Power Series}
\fgraph{Power Series in $x$}
\fgraph{Radius and Interval of Convergence}
\fgraph{Finding the Interval of Convergence}
\fgraph{Power Series in $x-x_0$}
\fgraph{Functions Defined by Power Series}
\subsection{Convergence of Taylor Series; Computational Methods}
\fgraph{The $n$th remainder}
\fgraph{Estimating the $n$th remainder}
\fgraph{Approximating Trigonometric Functions}
\fgraph{Roundoff and Truncation Error}
\fgraph{Approximating Exponential Functions}
\fgraph{Approximating Logarithms}
\fgraph{Approximating $\pi$}
\fgraph{Binomial Series}
\subsection{Differentiating and Integrating Power Series; Modeling with Taylor Series}
\fgraph{Differentiating Power Series}
\fgraph{Integrating Power Series}
\fgraph{Power Series Representations Must Be Taylor Series}
\fgraph{Some Practical Ways to Find Taylor Series}
\fgraph{Finding MacLaurin Series by Multiplication and Division}
\fgraph{Modeling Physical Laws with Taylor Series}
\section{Analytic Geometry in Calculus}
\subsection{Polar Coordinates}
\subsection{Tangent Lines and Arc Length for Parametric and Polar Curves}
\subsection{Area in Polar Coordinates}
\subsection{Conic Sections in Calculus}
\subsection{Conic Sections in Polar Coordinates}
\section{Three-Dimensional Space; Vectors}
\subsection{Rectangular Coordinates in 3-Space; Spheres; Cylindrical Surfaces}
\subsection{Vectors}
\subsection{Dot Product; Projections}
\subsection{Cross Product}
\subsection{Parametric Equations of Lines}
\subsection{Planes in 3-Space}
\subsection{Quadric Surfaces}
\subsection{Cylindrical and Spherical Coordinates}
\section{Vector-Valued Functions}
\subsection{Introduction to Vector-Valued Functions}
\subsection{Calculus of Vector-Valued Functions}
\subsection{Change of Parameters; Arc Length}
\subsection{Unit Tangent, Normal, and Binormal Vectors}
\subsection{Curvature}
\subsection{Motion Along a Curve}
\subsection{Kepler's Laws of Planetary Motion}
\section{Partial Derivatives}
\subsection{Functions of Two or More Variables}
\fgraph{Notation and Terminology}
\fgraph{Graphs of Functions of Two Variables}
\fgraph{Graphs of Functions of Two Variables Using Technology}
\fgraph{Level Curves}
\fgraph{Contour Plots Using Technology}
\fgraph{Level Surfaces}
\fgraph{Graphing Functions of Two Variables Using Technology}
\subsection{Limits and Continuity}
\fgraph{Open and Closed Sets}
\fgraph{Bounded Sets}
\fgraph{Limits Along Curves}
\fgraph{General Limits of Functions of Two Variables }
\fgraph{Properties of Limits}
\fgraph{Relationship Between General Limits and Limits Along Smooth Curves}
\fgraph{Continuity}
\fgraph{Limits at Points of Discontinuity}
\fgraph{Extension to Three Variables}
\subsection{Partial Derivatives}
\fgraph{Partial Derivatives of Functions of Two Variables}
\fgraph{Partial Derivatives Viewed as Rates of Change and Slopes}
\fgraph{Partial Derivative Notation}
\fgraph{Implicit Partial Differentiation}
\fgraph{Higher-Order Partial Derivatives}
\fgraph{The Wave Equation}
\fgraph{Partial Derivatives of Functions With More Than Two Variables}
\subsection{Differentiability and Chain Rules}
\fgraph{Differentiability of Functions of Two Variables}
\fgraph{Sufficient Conditions for Differentiability}
\fgraph{Equality of Mixed Partials}
\fgraph{Chain Rules}
\fgraph{Related Rates in Two Variables}
\subsection{Tangent Planes; Total Differentials for Functions of Two Variabls}
\fgraph{Tangent Planes}
\fgraph{The Geometric Signifiance of Differentiability}
\fgraph{Total Differentials}
\fgraph{Local Linear Approximation}
\fgraph{Approximations Using Total Differentials}
\subsection{Directional Derivatives and Gradients for Functions of Two Variables}
\fgraph{Directional Derivatives}
\fgraph{The Relationship Between Directional Derivatives and Partial Derivatives}
\fgraph{The Effect of Reversing Direction}
\fgraph{The Gradient}
\fgraph{Properties of the Gradient}
\fgraph{Gradients Are Normal to Level Curves}
\fgraph{An Application of Gradients}
\subsection{Differentiability, Directional Derivatives, and Gradients for Functions of Three or More Variables}
\fgraph{Differentiability}
\fgraph{Directional Derivatives and Gradients}
\fgraph{Gradient are Normal to Level Surfaces}
\fgraph{Using Gradients to Find Tangent Planes}
\fgraph{Using Gradients to Find Tangent Lines to Intersections of Surfaces}
\fgraph{Total Differentials}
\fgraph{Approximations Using Total Differentials}
\fgraph{Extensions to Functions of $n$ Variables}
\fgraph{Total Differentials}
\fgraph{Chain Rules}
\subsection{Maxima and Minima of Functions of Two Variables}
\fgraph{Extrema}
\fgraph{The Extreme-Value Theorem}
\fgraph{Finding Relative Extrema}
\fgraph{The Second Partials Test}
\fgraph{Finding Absolute Extrema on Closed and Bounded Sets}
\subsection{Lagrange Multipliers}
\fgraph{Extremum Problems with Constraints}
\fgraph{Lagrange Multipliers}
\fgraph{Three Variables and One Constraint}
% Multiple Integrals
\section{Multiple Integrals}
\subsection{Double Integrals}
\paragraph{Volume}
Double integrals can be used to compute volumes.\par\medskip
\paragraph{Definition of a Double Integral}
$$\iint_R f(x,y)\D A=\lim_{n\to+\infty}\sum_{k=1}^n f(x_k^\ast,y_k^\ast)\Delta A_k$$
\paragraph{Properties of Double Integrals} Similar rules for sums, differences, and products with constants.\par\medskip
Similar rules for subdivisions of areas.\par\medskip
\paragraph{Evaluating Double Integrals} Over a rectangle, $$\iint_Rf(x,y)\D A=\int_{y_\mathrm{min}}^{y_\mathrm{max}}\left(\int_{x_\mathrm{min}}^{x_\mathrm{max}}f(x,y)\D x\right)\D y$$
\subsection{Double Integrals over Nonrectangular Region}
\paragraph{Iterated Integrals with Nonconstant Limits of Integration}
$$\int_a^b\left(\int_{g_1(x)}^{g_2(x)}f(x,y)\D y\right)\D x$$
\paragraph{Double Integrals over Nonrectangular Regions}
Type I regions and Type II regions\par\medskip
\paragraph{Setting up Limits of Integration for Evaluating Double Integrals}\mbox{}\par\medskip
Example:
$$\iint_R xy\D A$$ between $y=\frac{x}{2}$ and $y=\sqrt{x}$, and $x=2$ and $x=4$\par\medskip
\paragraph{Reversing the Order of Integration}\mbox{}\par\medskip
Example:
$$\int_0^2\int_{y/2}^1e^{x^2}\D x$$
\fgraph{Area Calculated as a Double Integral}
\subsection{Double Integrals in Polar Coordinates}
\fgraph{Simple Polar Regions}
\fgraph{Double Integrals in Polar Coordinates}
\fgraph{Evaluating Polar Double Integrals}
\fgraph{Finding Areas Using Polar Double Integrals}
\fgraph{Converting Double Integral from Rectangular to Polar Coordinates}
\subsection{Parametric Surfaces; Surface Area}
\fgraph{Parametric Representation of Surfaces}
\fgraph{Representing Surfaces of Revolution Parametrically}
\fgraph{Vector-Valued Functions of Two Variables}
\fgraph{Partial Derivatives of Vector-Valued Functions}
\fgraph{Tangent Planes to Parametric Surfaces}
\fgraph{Surface Area of Parametric Surfaces}
\fgraph{Surface Area of Surfaces of the Form $z=f(x,y)$}
\subsection{Triple Integrals}
\fgraph{Definition of a Triple Integral}
\fgraph{Properties of Triple Integrals}
\fgraph{Evaluating Triple Integrals over Rectangular Boxes}
\fgraph{Evaluating Triple Integrals over More General Regions}
\fgraph{Volume Calculated as a Triple Integral}
\fgraph{Integration in Other Orders}
\subsection{Centroid, Center of Gravity, Theorem of Pappus}
\fgraph{Density of Lamina}
\fgraph{Mass of a Lamina}
\fgraph{Center of Gravity of a Lamina}
\fgraph{Centroids}
\fgraph{Center of Gravity and Centroid of a Solid}
\subsection{Triple Integrals in Cylindrical and Spherical Coordinates}
\fgraph{Triple Integrals in Cylindrical Coordinates}
\fgraph{Converting Triple Integrals from Rectangular to Cylindrical Coordinates}
\fgraph{Triple Integral in Spherical Coordinates}
\fgraph{Converting Triple Integral from Rectangular to Spherical Coordinates}
\subsection{Change of Variables in Multiple Integrals; Jacobians}
\fgraph{Change of Variable in a Single Integral}
\fgraph{Transformations of the Plane}
\fgraph{Jacobians in Two Variables}
\fgraph{Change of Variables in Double Integrals}
\fgraph{Change of Variables in Triple Integrals}
\section{Topics in Vector Calculus}
\subsection{Vector Fields}
\fgraph{Vector Fields}
\fgraph{Graphical Representations of Vector Fields}
\fgraph{A Compact Notation for Vector Fields}
\fgraph{Inverse-Square Fields}
\fgraph{Gradient Fields}
\fgraph{Conservative Fields and Potential Functions}
\fgraph{Divergence and Curl}
\fgraph{The $\nabla$ Operator}
\fgraph{The Laplacian $\nabla^2$}
\subsection{Line Integrals}
\fgraph{Line Integrals}
\fgraph{Evaluating Line Integrals}
\fgraph{Line Integrals in 3-Space}
\fgraph{Mass of a Wire as a Line Integral}
\fgraph{Arc Length as a Line Integral}
\fgraph{Line Integrals with Respect to $x$, $y$ and $z$}
\fgraph{Line Integrals along Piecewise Smooth Curves}
\fgraph{Change of Parameter in Line Integrals}
\fgraph{Reversing the Direction of Integration}
\fgraph{Work as a Line Integral}
\fgraph{A Method for Calculating Work}
\fgraph{Work Expressed in Scalar Form}
\subsection{Independence of Path; Conservative Vector Fields}
\fgraph{Work Integrals}
\fgraph{Independence of Path}
\fgraph{The Fundamental Theorem of Work Integrals}
\fgraph{Work Integrals Along Closed Paths}
\fgraph{A Test for Conservative Vector Fields}
\fgraph{Conservative Vector Fields in 3-Space}
\fgraph{Conservative of Energy}
\subsection{Green's Theorem}
\fgraph{Green's Theorem}
\fgraph{A Notation for Line Integrals Around Simple Closed Curves}
\fgraph{Finding Work Using Green's Theorem}
\fgraph{Finding Areas Using Green's Theorem}
\fgraph{Green's Theorem for Multiply Connected Regions}
\subsection{Surface Integral}
\fgraph{Definition of a Surface Integral}
\fgraph{Evaluating Surface Integrals}
\fgraph{Surface Integrals over $z=g(x,y)$, $y=g(x,z)$ and $x=g(y,z)$}
\fgraph{Mass of a Curved Lamina as a Surface Integral}
\fgraph{Surface Area as a Surface Integral}
\subsection{Application of Surface Integras; Flux}
\fgraph{Flow Fields}
\fgraph{Oriented Surfaces}
\fgraph{Orientation of a Smooth Parametric Surface}
\fgraph{Evaluating Flux Integrals}
\fgraph{Orientation of Nonparametric Surfaces}
\subsection{The Divergence Theorem}
\fgraph{Orientation of Piecewise Smooth Closed Surfaces}
\fgraph{The Divergence Theorem}
\fgraph{Using the Divergence Theorem to Find Flux}
\fgraph{Divergence Viewed as Flux Density}
\fgraph{Sources and Sinks}
\fgraph{Gauss's Law for Inverse-Square Fields}
\fgraph{Gauss's Law in Eletrostatics}
\subsection{Stokes' Theorem}
\fgraph{Relative Orientation of Curves and Surfaces}
\fgraph{Stoke's Theorem}
\fgraph{Using Stoke's Theorem to Calculate Work}
\fgraph{Relationship Between Green's Theorem ad Stoke's Theorem}
\fgraph{Curl Viewed as Circulation}
\end{document}
